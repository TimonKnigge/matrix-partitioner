\section{Experimental results}
\label{sec:experiments}
To test the capabilities of the new exact matrix partitioner MP
and to compare it with MondriaanOpt, we performed numerical experiments
on a subset of small and medium-sized test matrices from the SuiteSparse
Matrix Collection (formerly known as the University
of Florida Sparse Matrix Collection~\cite{davis11}).
We chose as test set the subset of all sparse matrices with at most 100,000 nonzeros,
which contains 1602 matrices. (After having removed five matrices with duplicate names:
	\texttt{Pothen/\-barth}, \texttt{Pothen/\-barth4}, \texttt{Mesz\-aros/\-fxm3\_6},
    \texttt{Boeing/\-nasa\-1824}, \texttt{Pajek/\-foot\-ball}.
	We kept the \texttt{barth}, \texttt{barth4}, and \texttt{nasa1824} matrices
          from the \texttt{Nasa} subcollection, and  \texttt{Andri\-anov/\-fxm3\_6} and
          \texttt{New\-man/\-foot\-ball}.)
Currently, the whole collection contains 2833 matrices.
To keep the total CPU time used within reasonable bounds, we allotted
a maximum of 24 hours of CPU time to each partitioning run.

All computations were carried out on thin nodes with 24 cores of the Dutch national supercomputer
Cartesius at  SURFsara in Amsterdam, with a core clock speed of 2.4 GHz (for Intel Ivy Bridge E5-2695 v2  CPUs)
or 2.6 GHz (for Haswell Bridge E5-2690 v3 CPUs). The memory for each 
 thin node is 64 GB. Each batch of 24 jobs is assigned to a node by a runtime scheduler,
which may lead to different types of CPUs being used, causing a slight inconsistency in our timings.
All MP runs were carried out on the slower Ivy Bridge nodes and all MondriaanOpt nodes on the faster
Haswell nodes.
To normalize the timings, we performed calibration runs for the 40 longest running matrices
using MP on both types of nodes of the Cartesius computer, and based on the geometric mean for this set,
we found that the Haswell nodes 
are a factor of  $\alpha= 1.1782$ times faster than the Ivy Bridge nodes.
When comparing MP with MondriaanOpt, we multiplied the MondriaanOpt timings
by this factor. Note that this still gives MondriaanOpt a small advantage in that it can solve more problems
within 24 hours on the faster Haswell nodes, than it could do on the slower Ivy Bridge nodes.

We implemented the branch-and-bound algorithm from section~\ref{sec:opt} in our new program
MP\footnote{MP is available from \url{https://github.com/TimonKnigge/matrix-partitioner}}. The
implementation was done in C++14, and the final program was compiled with GNU GCC Version 7.1.0
with the \texttt{-O2} flag.
We chose a value of $\epsilon=0.03$ in Equation~(\ref{eqn:imbal}), which is a common value
allowing a trade-off between load imbalance and communication volume. 
As a result of our numerical experiments, we may divide the matrices into three groups: (i) a group of 368 matrices which could be solved
by both programs, MP and MondriaanOpt. We use these matrices to compare the speed of the two programs
and to verify their correctness; (ii) a group of 471 matrices which could only be solved by MP;
(iii) a group of 763 matrices which could not be solved by either program.
All matrices that could be solved by MondriaanOpt within 24 hours could also be solved by 
MP within 24 hours.

For the 368 matrices that could be solved by both programs, all optimal volumes computed are identical
for the two programs,
which we take as an independent mutual confirmation of their correctness. We have taken great care
in developing the programs to make them understandable and to document them well,
to support our claim that they compute exact, optimal solutions.
Both programs are available as open-source software
and are open to inspection for correctness.
The two programs do not necessarily compute the same solution, as there may be several
optimal solutions. The optimal volume, however, is of course unique.

The program MP is faster than MondriaanOpt in 306 cases (83\% of the cases).
In 25 cases (6.7\%), it performs equally well, of which 22 cases with both programs needing exactly
1 second (our clock resolution), and having a volume of 0.
In 37 cases (10\%), MondriaanOpt is faster, of which 31 cases with volume 0.
For volume 0, the sparse matrix can be split into several connected components
(when viewed as a graph) of suitable sizes. This situation is easy to handle
and it is quickly discovered by both programs. Note that in this comparison of individual matrices
we did not normalize by the factor $\alpha$.
The geometric average of the ratio $T_{\mathrm{MP}} / T_{\mathrm{Opt}}$
between the time of MP and the normalized time of MondriaanOpt is $0.0855 / \alpha = 0.0726$,
meaning that MP is about 13.8 times faster
than MondriaanOpt.
This average is based on 286 matrices that could be solved by both programs and for which $T_{\mathrm{MP}}, T_{\mathrm{Opt}}  \geq 1$.

Table~\ref{tbl:matrices} shows the hardest cases that MP could still
solve within our self-imposed time limit.
These are in fact the matrices that take between 4 and 24 hours to be solved.
Note that there is no simple parameter that characterises the most difficult  matrices
for bipartitioning. Still we can say 
that the hardest solvable matrices usually have a communication volume of 10 or more,
with the exception
of the matrix \texttt{mhd4800b}, which has a low volume of 2.
Furthermore, they also have at least 1000 nonzeros, with two exceptions, \texttt{ch4-4-b2}
and \texttt{GD97\_a}. This particular top-40 set is our challenge to future exact partitioners.


\begin{table*}[p]
\caption{The top-40 of matrices with the longest computation time needed by
the matrix partitioner MP. Given are the matrix name, number of rows, columns, and nonzeros,
the optimal communication volume for the choice $\epsilon=0.03$,
and the CPU time (in s) needed for computing 
an optimal solution.}
\label{tbl:matrices}

\begin{center}
\begin{tabular}{lrrrrr}
\hline
Name&\multicolumn{1}{c}{$m$}&
\multicolumn{1}{c}{$n$}&
\multicolumn{1}{c}{$nz$}&
\multicolumn{1}{c}{$V_{\mathrm{MP}}$}&
\multicolumn{1}{c}{$T_{\mathrm{MP}}$}\\
\hline
\hline
\texttt{mhd1280a}&1280&1280&47906&44&15382\\
\texttt{reorientation\_4}&2717&2717&33630&14&15839\\
\texttt{lp\_stocfor3}&16675&23541&76473&14&16531\\
\texttt{bp\_1000}&822&822&4661&35&17170\\
\texttt{west0479}&479&479&1910&33&17858\\
\texttt{ch4-4-b2}&96&72&288&24&18544\\
\texttt{DK01R}&903&903&11766&20&18569\\
\texttt{celegansneural}&294&270&2345&57&19061\\
\texttt{c-28}&4598&4598&30590&10&19257\\
\texttt{orbitRaising\_4}&915&915&7790&16&22536\\
\texttt{circuit204}&1020&1020&5883&41&22812\\
\texttt{GD97\_a}&84&84&332&24&23357\\
\texttt{lp\_modszk1}&686&1620&3168&34&24476\\
\texttt{pcb1000}&1565&2820&20463&40&26657\\
\texttt{mhd4800b}&4800&4800&27520&2&29574\\
\texttt{lp\_grow22}&440&946&8252&20&30335\\
\texttt{bayer02}&13935&13935&63679&27&30674\\
\texttt{kineticBatchReactor\_5}&7641&7641&80767&18&31632\\
\texttt{can\_256}&256&256&2916&43&32977\\
\texttt{dynamicSoaringProblem\_4}&3191&3191&36516&22&34058\\
\texttt{qiulp}&1192&1900&4492&40&34828\\
\texttt{lp\_bnl1}&642&1586&5532&47&36413\\
\texttt{Hamrle2}&5952&5952&22162&16&36965\\
\texttt{ex21}&656&656&19144&62&37236\\
\texttt{fs\_541\_1}&541&541&4285&37&38634\\
\texttt{fs\_541\_4}&541&541&4285&37&38750\\
\texttt{fs\_541\_3}&541&541&4285&37&38836\\
\texttt{fs\_541\_2}&541&541&4285&37&39259\\
\texttt{c-29}&5033&5033&43731&28&40217\\
\texttt{bp\_600}&822&822&4172&33&40328\\
\texttt{model1}&362&798&3028&46&43785\\
\texttt{ncvxqp9}&16554&16554&54040&30&51673\\
\texttt{de063157}&936&1656&5119&36&56101\\
\texttt{kineticBatchReactor\_4}&7105&7105&74869&18&58461\\
\texttt{kineticBatchReactor\_9}&8115&8115&86183&18&62664\\
\texttt{lp\_sctap2}&1090&2500&7334&40&67631\\
\texttt{can\_229}&229&229&1777&38&76406\\
\texttt{iiasa}&669&3639&7317&14&76561\\
\texttt{lpi\_pilot4i}&410&1123&5264&47&79647\\
\texttt{lp\_pilot4}&410&1123&5264&47&79910\\
\hline
\end{tabular}
\end{center}
\end{table*}
