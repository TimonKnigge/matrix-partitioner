\documentclass{article}

\usepackage[a4paper, margin=1.37in]{geometry}
\usepackage[english]{babel}
\usepackage[utf8]{inputenc}
\usepackage{amsmath}
\usepackage{amsfonts}
\usepackage{amsthm}
\usepackage{amssymb}
\usepackage{fancyhdr}
\usepackage{xifthen}
\usepackage{tabularx,lipsum,environ,amsmath,amssymb}
\usepackage{cite}
\usepackage[T1]{fontenc}
\usepackage{todonotes}
\usepackage{hyperref}
\usepackage{tikz}
\usepackage{aliascnt}
\usepackage{tikz-qtree,tikz-qtree-compat}

\usetikzlibrary{shapes,arrows,matrix,positioning,calc,positioning,automata,shadows,fit}

% Matrix drawing
\newcommand{\msize}{2em}
\tikzstyle{redblock} = [draw, fill=red, rectangle, minimum width=\msize, minimum height=\msize]
\tikzstyle{blueblock} = [draw, fill=blue, rectangle, minimum width=\msize, minimum height=\msize]
\tikzstyle{grayblock} = [draw, fill=gray, rectangle, minimum width=\msize, minimum height=\msize]
\tikzstyle{whiteblock} = [draw, fill=white, rectangle, minimum width=\msize, minimum height=\msize]

\newcommand{\redb}{\node[redblock] {};}
\newcommand{\blueb}{\node[blueblock] {};}
\newcommand{\grayb}{\node[grayblock] {};}
\newcommand{\whiteb}{\node[whiteblock] {};}
	
%	\begin{tikzpicture}
%	\matrix[matrix of nodes] {
%		\redb  &\redb  &\whiteb&\whiteb\\
%		\redb  &\redb  &\whiteb&\blueb \\
%		\whiteb&\whiteb&\blueb &\blueb \\
%		\whiteb&\whiteb&\blueb &\whiteb\\
%	};
%	\end{tikzpicture}

% NP PROBLEM
\makeatletter
\newcommand{\problemtitle}[1]{\gdef\@problemtitle{#1}}% Store problem title
\newcommand{\probleminput}[1]{\gdef\@probleminput{#1}}% Store problem input
\newcommand{\problemquestion}[1]{\gdef\@problemquestion{#1}}% Store problem question
\NewEnviron{problem}{
  \problemtitle{}\probleminput{}\problemquestion{}% Default input is empty
  \BODY% Parse input
  \par\addvspace{.5\baselineskip}
  \noindent
  \begin{tabularx}{\textwidth}{@{\hspace{\parindent}} l X c}
    \multicolumn{2}{@{\hspace{\parindent}}l}{\textsc{\@problemtitle}} \\% Title
    \textbf{Input:} & \@probleminput \\% Input
    \textbf{Question:} & \@problemquestion% Question
  \end{tabularx}
  \par\addvspace{.5\baselineskip}
}

% THEOREM ENVIRONMENTS
\newtheorem{theorem}{Theorem}[section]

\newaliascnt{lemma}{theorem}
\newtheorem{lemma}[lemma]{Lemma}
\aliascntresetthe{lemma}
\providecommand*{\lemmaautorefname}{Lemma}

\newaliascnt{proposition}{theorem}
\newtheorem{proposition}[proposition]{Proposition}
\aliascntresetthe{proposition}
\providecommand*{\propositionautorefname}{Proposition}

\newaliascnt{corollary}{theorem}
\newtheorem{corollary}[corollary]{Corollary}
\aliascntresetthe{corollary}
\providecommand*{\corollaryautorefname}{Corollary}

\newaliascnt{definition}{theorem}
\newtheorem{definition}[definition]{Definition}
\aliascntresetthe{definition}
\providecommand*{\definitionautorefname}{Definition}

%\newtheorem{lemma}[theorem]{Lemma}
%\newtheorem{proposition}[theorem]{Proposition}
%\newtheorem{corollary}[theorem]{Corollary}
%\newtheorem{definition}[theorem]{Definition}

% NP Complete
\newcommand{\NP}{$\mathcal{NP}$}
\newcommand{\NP}{$\mathcal{P}$}

% Do not split footnotes over multiple pages
\interfootnotelinepenalty=10000



% Add commands here

% Decision problems
\newcommand{\mbpt}{\textsc{Matrix \allowbreak Bipartition} }
\newcommand{\gb}{\textsc{Graph \allowbreak Bisection} }
\newcommand{\geb}{\textsc{Graph \allowbreak Edge-\allowbreak Bisection} }
\newcommand{\bgeb}{
	\textsc{Bipartite \allowbreak Graph \allowbreak Edge-\allowbreak Bisection}
 }
\newcommand{\sgeb}{\textsc{GEB} }
\newcommand{\sgb}{\textsc{GB} }

% /commands

\begin{document}

	\section{Hardness results}
	In this section we will formally analyze the matrix partitioning
	problem and prove that it is \NP-Complete, even if we fix the
	the number of processors to $k = 2$. We will assume we are looking
	for a perfect partitioning, i.e. with imbalance parameter $\epsilon = 0$.

	\subsection{Preliminaries}
	\label{1-preliminaries}
	To begin, let us define a formal decision-variant of the matrix
	partitioning problem for $k = 2$, based on the optimization
	variant described in \autoref{introductory-section} where the goal
	is to minimize the total communication volume. We formulate our
	decision problems in the style of \cite{npbook}.

	\begin{problem}
		\problemtitle{\mbpt}
		\probleminput{A $n\times m$ matrix $M$, whose nonzeros are precisely
			indexed by the set
			$Z \subseteq \{\, 1, \dots, n \,\} \times
				\{\, 1, \dots, m \,\}$, and an integer $V$, the required
			maximum volume.}
		\problemquestion{Does there exist a disjoint partitioning of $Z$
			into $Z_1 \cup Z_2$ such that $|Z_1| = |Z_2|$ and volume
			$VOL(Z_1, Z_2) \leq V$?}
	\end{problem}

	Here $VOL(Z_1, Z_2)$ counts the number of rows and columns that
	have nonzeros in $Z_1$ and $Z_2$, as before.

	When thinking about the matrix bipartitioning problem, it is helpful
	to reformulate it in terms of graphs. Given an $n \times m$ matrix $M$
	we can define a bipartite adjacency graph $G(M) = (V(M), E(M))$ with
	$n$ vertices representing the rows of $M$, and $m$ vertices representing
	the columns, where a row vertex $r$ and a column vertex $c$ are connected
	if and only if $M_{rc}$ is nonzero.

	This equivalence extends to the partitioning problem. A bipartitioning of
	the nonzeros of $M$ corresponds to a bipartitioning of the edges of $G(M)$,
	and the rows and columns contributing to the final volume are correspond
	precisely to the vertices with edges in both sides of the partition. See
	also \autoref{figure-for-graph-equiv}.

	\todo{Add image of graph equiv.}
	\todo{Continue by describing equivalence of coloring and partitioning}

	It should be clear this procedure is also reversible, i.e. for any
	bipartite graph $G$ on $n$ and $m$ vertices,
	we can construct a corresponding matrix $M$ of size $n \times m$ which has
	nonzeros precisely for the vertices that are connected in $G$. While the
	nonzero entries of this matrix can be any value, the associated nonzero
	pattern is uniquely determined by the edges of $G$.

	To this end we define an equivalent bipartitioning problem on graphs that
	we will base our reduction on:

	\begin{problem}
		\problemtitle{\geb}
		\probleminput{Given a graph
			$G = (V, E)$ and an integer
			$M$.}
		\problemquestion{Does there exist a disjoint partitioning of $E$
			into $E_1 \cup E_2$ such that $|E_1| = |E_2|$ and
			$\big|\big(\bigcup E_1 \big) \cap
				\big(\bigcup E_2 \big)\big| \leq M$?\footnote{
			Writing an edge as the set $\{ u, v\} \subset V$, then
			$\bigcup E_1$ gives precisely the set of vertices touched
			by $E_1$.}}
	\end{problem}

	We call a vertex with adjacent edges from both partitions `cut'. The
	goal is to minimize the amount of cut vertices. Additionally, when we
	explicitly need the partitioning/coloring of $E$, we will write it 
	as a map $\pi : E \to \{0,1\}$.

	\subsection{\geb is \NP-Complete}
	We will do a reduction from \gb which was proven \NP-Complete in
	\cite{gb-np-complete} under the name \textsc{Minimum \allowbreak
	cut \allowbreak into \allowbreak equal-\allowbreak sized \allowbreak
	subsets}.

	\begin{problem}
		\problemtitle{\gb}
		\probleminput{A graph $G = (V, E)$, an integer $M$.}
		\problemquestion{Does there exist a disjoint partition of $V$
			into $V_1 \cup V_2$ with $|V_1| = |V_2|2$ and such that
			$|\{\, \{u, v\} \,\mid\, u \in V_1, v\in V_2 \,\}| \leq M$?}
	\end{problem}

	Analogously to \geb, an edge with endpoints in both sides of the
	partition is called `cut', the goal is to minimize the amount of
	cut edges. We similarly write a partitioning/coloring of $V$ as
	a map $\tau : V \to \{0, 1\}$.

	Let us give a sketch of our proof strategy:
	given an instance $(G, M)$ of the \gb problem, we
	will build a new graph $G'$, whose optimal solution under the 
	\geb problem will give us an optimal solution under \gb on $G$.

	Specifically, for each vertex $u \in V$ we create a clique $K_u$ of size
	$S$ (specified later). For every edge $\{u, v\} \in E$, we merge
	two vertices in the cliques $K_u$ and $K_v$ together into a single
	vertex. We then solve the \geb on the resulting graph $(V', E')$, and
	translate the resulting coloring of its edges into a coloring of the
	vertices of $(V, E)$. (Here, the colors correspond to sides in the
	partition.) We note that if we color each clique in $(V', E')$
	monochromatically with the color of its vertex in $(V, E)$, then
	an edge between two differently colored vertices in $(V, E)$ will
	correspond precisely with a vertex shared by two differently colored
	cliques in $(V', E')$.
	An example is given in \autoref{figure-graph-clique-conversion}.

	\begin{figure}[h]
		% Fast clique drawing
\newcommand\single[3]{ % #1=labels, $2= n=number of nodes
	\foreach \x in {1,...,#2}{
		\pgfmathsetmacro{\ang}{360/#2}
		\pgfmathparse{(\x-1)*\ang}
		\node[shape=circle,fill=black,draw=black] (#1-\x)
			at (\pgfmathresult:1cm) {};
	}
	\foreach \x [count=\xi from 1] in {1,...,#2}{
		\foreach \y in {\x,...,#2}{
			\path (#1-\xi) edge[color=#3,line width=1.75pt] (#1-\y);
		}
	}
}
\begin{tikzpicture}%[thick,scale=0.8, every node/.style={transform shape}]

	% Arrows giving direction of conversion
	\begin{scope}
		\node at (4, 0){\huge$\rightarrow$};
	\end{scope}

	\begin{scope}
		\node at (8, -2.5){\huge$\downarrow$};
	\end{scope}

	\begin{scope}
		\node at (4, -5){\huge$\leftarrow$};
	\end{scope}

	% Initial graph
	\begin{scope}
		\node[shape=circle,fill=black,draw=black,minimum size=13pt] (1)
				at ($(-0.5,{sqrt(3)*-0.5})$) {};
		\node[shape=circle,fill=black,draw=black,minimum size=13pt] (2)
				at ($(-0.5,{sqrt(3)*0.5})$) {};
		\node[shape=circle,fill=black,draw=black,minimum size=13pt] (3)
				at (1, 0){};
		\node[shape=circle,fill=black,draw=black,minimum size=13pt] (4)
				at ($({1+sqrt(2)}, 0)$) {};

		\draw[line width=1.5pt] (1) -- (2) -- (3) -- (1);
		\draw[line width=1.5pt] (3) -- (4);
	\end{scope}

	% Uncolored cliques
	\begin{scope}[local bounding box=scope1,shift={(7, 0)}]
		\foreach \s[count=\si from 0] in {0,120,240}{
			\begin{scope}[shift={($(\s:1.1414)$)}, rotate=\s]
				\single{\si}{7}{black};
			\end{scope}
		}
		\begin{scope}[shift={(3.1415, 0)}, rotate=180]
			\single{4}{7}{black};
		\end{scope}
	\end{scope}

	% Colored cliques
	\begin{scope}[local bounding box=scope1,shift={(7, -5)}]
		\foreach \s[count=\si from 0] in {0,120,240}{
			\begin{scope}[shift={($(\s:1.1414)$)}, rotate=\s]
				\ifthenelse{\equal{\si}{0}}
					{\single{\si}{7}{red};}
					{\single{\si}{7}{blue};}
			\end{scope}
		}
		\begin{scope}[shift={(3.1415, 0)}, rotate=180]
			\single{4}{7}{red};
		\end{scope}
	\end{scope}

	% Final graph
	\begin{scope}[shift={(0, -5)}]
		\node[shape=circle,fill=blue,draw=blue,minimum size=13pt] (1)
				at ($(-0.5,{sqrt(3)*-0.5})$) {};
		\node[shape=circle,fill=blue,draw=blue,minimum size=13pt] (2)
				at ($(-0.5,{sqrt(3)*0.5})$) {};
		\node[shape=circle,fill=red,draw=red,minimum size=13pt] (3)
				at (1, 0){};
		\node[shape=circle,fill=red,draw=red,minimum size=13pt] (4)
				at ($({1+sqrt(2)}, 0)$) {};

		\draw[line width=1.5pt] (1) -- (2) -- (3) -- (1);
		\draw[line width=1.5pt] (3) -- (4);
	\end{scope}
\end{tikzpicture}

		\centering
		\caption{Solving \gb using \geb.}
		\label{figure-graph-clique-conversion}
	\end{figure}

	We note that
	the \geb problem gives us no guarantee that each clique is colored
	monochromatically, but we will work around this later.

	Formally, let us define the \textit{clique expansion} of a graph as
	follows:

	\begin{definition}
		\label{clique-expansion}
		Given a graph $G = (V, E)$,
		let $S = 4 + 2|V|\binom{|E|}{2}$.
		We define its clique expansion expansion
		$\mathcal{K}(G) = (V', E')$ as first taking a disjoint union of $|V|$
		copies of the complete graph on $S$ vertices $K_S$. Then, labelling the
		edges in $E$ as $e_1, e_2, \dots, e_{|E|}$, for each edge
		$e_i = \{u, v\}$
		we merge the $i^{th}$ vertex of the $K_S$ representing $u$ with the
		$i^{th}$ vertex of the $K_S$ representing $v$.
	\end{definition}

	Here, the clique size $S$ will allow us to prove several
	useful propositions later. Notice that by construction, each pair of
	cliques is merged at most once, each time in a previously unmerged vertex.
	As a consequence, while cliques share vertices, they do not share edges.

	Additionally, throughout this section we will make a slight abuse of
	terminology. A clique usually (and up until now) refers to any collection
	of pairwise connected vertices. However, from now on, when we talk about
	`cliques' in $\mathcal{K}(G)$ we will be referring specifically to the
	maximal cliques corresponding to vertices, i.e. the cliques
	$\{\, K_u \,\mid\, u \in V \,\}$ in $\mathcal{K}(G)$.

	\begin{proposition}
		\label{poly-size}
		The size of $\mathcal{K}(G) = (V', E')$ is polynomial in the size of
		$G = (V, E)$.
	\end{proposition}
	\begin{proof}
		By construction, each vertex in $V$ induces a subgraph with
		$O(|V||E|^2)$
		vertices and $O(|V|^2|E|^4)$ edges. After merging, the graph will only
		become smaller. So $|V'|$ is $O(|V|^2 |E|^2)$ and $|E'|$ is
		$O(|V|^3|E|^4)$.
	\end{proof}

	So $\mathcal{K}(G)$ is of polynomial size and we can solve the \geb problem
	on it. We now want to show that both problems have equal optimal solutions
	(in the sense of an optimization problem, not a decision problem). For
	a graph $G=(V, E)$, let $\sgb(G)$ denote the optimal solution of the \gb
	problem on $G$, and let $\sgeb(\mathcal{K}(G))$ denote the optimal solution
	of the \geb problem on its \textit{clique expansion} $\mathcal{K}(G)$.

	\begin{proposition}
		\label{geb-leq-gb}
		For any graph $G$ we have $\sgeb(\mathcal{K}(G)) \leq \sgb(G)$.
	\end{proposition}
	\begin{proof}
		Consider any valid equipartitioning $\tau$ of $G = (V, E)$ of cost
		$C$ (that is, there are
		exactly $C$ edges $\{u, v\}$ with $\tau(u) \neq \tau(v)$).
		For any vertex $u$,
		color all edges in the corresponding clique $K_u$ in $\mathcal{K}(G)
			= (V', E')$ with the same color, i.e. for any edge $e$ in $K_u$
		let $\pi(e) := \tau(u)$.

		Since each clique $K_u$ in $\mathcal{K}(G)$ has the same number of
		edges ($\binom{S}{2}$), and by
		assumption the partitioning of $V$ is into two equal sized subsets, the
		strategy described above partitions $E'$ into two equal sized subsets.

		Now let $s$ be a vertex in $\mathcal{K}(G)$. If $s$ is contained in
		only one clique, it is not cut,
		since we color the edges of
		each $K_u$ monochromatically. If $s$ is shared by two
		cliques $K_u$ and $K_v$, then $s$ corresponds to the edge
		$e = \{u, v\} \in E$, and
		we can see that this vertex is cut by $\pi$
		if and only if $e$ is cut by $\tau$ (since $K_u$ and
		$K_v$ are colored like $u$ and $v$ respectively).
		Since by construction,
		each vertex is in at most two cliques, there is no ambiguity.

		So the number of cut vertices in the induced partitioning $\pi$ of the
		edges of
		$\mathcal{K}(G)$ is exactly the number of cut edges in the original
		partitioning $\pi$ of the vertices of $G$. We can then minimize
		over all valid partitionings of $G$ to achieve the desired inequality.
	\end{proof}

	Unfortunately the converse is harder to prove since we cannot guarantee
	that an optimal partitioning of $\mathcal{K}(G)$ colors each clique
	monochromatically. It turns
	out however, that we can still deterministically associate a color with
	each clique, provided we have an \textbf{optimal coloring}.

	\begin{definition}
		Let $K$ be a clique and suppose we have a coloring of its edges. The
		dominating color of $K$ is a color $c$ such that there exists a vertex
		in $K$ with all of its adjacent edges colored $c$.
	\end{definition}

	While this property is clearly not well-defined in general, it is for our
	restricted case:

	\begin{lemma}
		\label{dom-col}
		Given a graph $G = (V, E)$ and an optimal partitioning $\pi$
		of the edges of its clique
		expansion $\mathcal{K}(G)$. Then each clique $K_u$ in $\mathcal{K}(G)$
		has a well-defined dominating color.
	\end{lemma}
	\begin{proof}
		Fix a clique $K_u$ in $\mathcal{K}(G)$. We need to prove existence and
		uniqueness of its dominating color.

		First, uniqueness is trivial. To the contrary, assume there are two
		vertices $r, b$ in $K_u$ such that $r$ has only red edges adjacent,
		and $b$ only blue edges. Since $K_u$ is a clique, the edge $\{r, b\}$
		exists, which must be both red and blue, a contradiction.

		As for existence, assume to the contrary that every vertex in $K_u$
		has both blue and red edges adjacent. But this means each vertex in
		$K_u$ is cut by $\pi$ the partitioning, and so the cost of
		this partitioning of $\mathcal{K}(G)$ is at least the clique size
		$S = 4+2|V|\binom{|E|}{2}$. One may verify that for any graph,
		$S > |E|$. However, $|E|$ is a trivial upperbound on the \gb problem
		on $(V, E)$ (in which we cut \textit{every} edge in $E$),
		which, by \autoref{geb-leq-gb} is an upperbound on the optimal
		partitioning of the edges of $\mathcal{K}(G)$. Since we assumed our
		partitioning $\pi$ is optimal, i.e. has cut size
		exactly equal to $\sgeb(\mathcal{K}(G))$, this implies that

		$$|E| \geq \sgb(G) \geq \sgeb(\mathcal{K}(G)) \geq S > |E|$$

		which is a contradiction, so there must exist a vertex that only has
		adjacent edges of a single color (in fact, we have shown $K_u$ must
		contain at least $S - |E| > 0$ such vertices).
	\end{proof}

	In addition to the above, we would like to note in particular that by
	definition, if $K_u$ has dominating color $c$, then any vertex in $K_u$
	has at least one adjacent edge with color $c$.

	Our strategy should now be obvious. We will color vertices in $G$ by
	the dominating color of their cliques in an optimal partitioning of
	$\mathcal{K}(G)$.

	\begin{proposition}
		\label{gb-leq-geb}
		For any graph $G$ we have $\sgb(G) \leq \sgeb(\mathcal{K}(G))$.
	\end{proposition}
	\begin{proof}
		Fix any optimal partitioning $\pi$ of $\mathcal{K}(G) = (V', E')$, and
		let $\tau$ color each vertex $u$ in $G = (V, E)$ with the dominating
		color of its associated clique $K_u$ in $\mathcal{K}(G)$. We would
		like to
		prove two things about this partitioning $\tau$:
		that the number of cut edges in $G$ is no more than the number of
		cut vertices in $\mathcal{K}(G)$, and that it equipartitions $V$.


		\begin{lemma}
			\label{gb-leq-geb-lemma-1}
			The number of edges cut by $\tau$ in $G$ is no more than the number
			of vertices cut by $\pi$ in $\mathcal{K}(G)$.
		\end{lemma}
		\begin{proof}
			Suppose $\tau$ cuts edge $e_i = \{u, v\} \in E$, that is,
			$\tau(u) \neq \tau(v)$. This means the dominating colors of
			of $K_u$ and $K_v$ are different, say without loss of generality
			that $K_u$ is red and $K_v$ is blue. Hence, the vertex $s$ in
			$\mathcal{K}(G)$ that
			corresponds to $e_i$, which we obtained during construction
			by merging the $i^{th}$
			vertex of $K_u$ with the $i^{th}$ vertex of $K_v$, must have
			red edges adjacent, because it is contained in $K_u$, and blue
			edges, because it is contained in $K_v$.

			So $\pi$ cuts $s$.
			Since for every edge $e_j \in E$ we merged different
			vertices (specifically, for $e_j$ we used the $j^{th}$ vertex of
			the two cliques), each edge in $E$ cut by $\tau$ has a unique
			corresponding vertex $s$ in $\mathcal{K}(G)$ cut by $\pi$,
			proving the lemma.
			\todo{In mijn scriptie deed ik dit in twee stappen door eerst te
				bewijzen dat je gratis elke clique kan herkleuren naar zijn
				dominante kleur, en dan de correspondentie $K_u \to u$ kan
				gebruiken. Dit is iets korter, ik hoop dat het nog duidelijk
				is.}
		\end{proof}
		\begin{lemma}
			\label{gb-leq-geb-lemma-2}
			$\tau$ equipartitions $V$.
		\end{lemma}
		\begin{proof}
			Equivalently, we would like to prove that our optimal partitioning
			$\pi$ of $\mathcal{K}(G)$ contains as many cliques with red as
			their dominating color as it contains cliques with blue as their
			dominating color.

			Let $r,b \geq 0$, $r+b=|V|$ count these quantities, assuming
			without loss of generality that $r \geq b$. Now consider a lower
			bound on the number of red edges in a clique in $\mathcal{K}(G)$.
			In each red clique we have at most
			$|E|$ cut vertices (since we assumed our partitioning was optimal,
			as in the proof of \autoref{dom-col}) (in fact,
			across all cliques there are at most $|E|$ cut vertices, but for a
			lower bound this will suffice), and the edges between two such
			vertices may be blue, but none of the other $S - |E|$ vertices are
			cut so, so all other edges should be red, and a lower
			bound on the number of red edges in $\pi$ is:

			$$r\Big( \binom{S}{2} - \binom{|E|}{2}\Big)$$

			Similarly, we can find an upper bound for the number of blue edges
			by the following reasoning: we color each blue-dominated clique
			entirely blue, and as many edges as possible in each red-dominated
			clique (at most $\binom{|E|}{2}$, as before). This gives as
			upperbound:

			$$b\binom{S}{2} + r\binom{|E|}{2}$$

			But since $\pi$ was an optimal \textit{equi}partitioning of the
			edges of $\mathcal{K}(G)$, certainly the lower bound on the amount
			of red edges must be smaller than or equal to the upper bound on
			the amount of blue edges:

			$$r\Big( \binom{S}{2} - \binom{|E|}{2}\Big)
				\leq b\binom{S}{2} + r\binom{|E|}{2}$$

			Reordering terms:

			\begin{equation}
				\label{ineq-pf}
				(r-b)\binom{S}{2} \leq 2r\binom{|E|}{2}
			\end{equation}

			Recall that we set $S = 4 + 2|V|\binom{|E|}{2}$. Since $S \geq 4$
			we have $S \leq \binom{S}{2}$. Also, clearly $r \leq |V|$. So if
			\autoref{ineq-pf} holds, then certainly the following holds:

			$$(r-b)S \leq 2|V|\binom{|E|}{2}$$

			Substituting $S$ and rewriting we get:

			$$4(r-b) + 2(r-b-1)|V|\binom{|E|}{2} \leq 0$$
 
			Since $r \geq b$, clearly this can only hold if $r=b$.
		\end{proof}

		Together \autoref{gb-leq-geb-lemma-1} and \autoref{gb-leq-geb-lemma-2}
		imply that we can turn any optimal solution to the \geb problem on
		$\mathcal{K}(G)$ into a solution of equal value to the \gb problem on
		$G$, proving \autoref{gb-leq-geb}.
	\end{proof}

	We can now conclude:

	\begin{theorem}
		\geb is \NP-Complete.
	\end{theorem}
	\begin{proof}
		We claim that $\gb \leq_{\mathcal{P}} \geb$.
		For a given instance of \gb $(G, M)$, by \autoref{geb-leq-gb} and
		\autoref{gb-leq-geb} we know
		$\sgb(G) \leq M$ if and only if $\sgeb(\mathcal{K}(G)) \leq M$. So
		if we can solve \geb on $\mathcal{K}(G)$ (which has size polynomial
		in the size of $G$, by \autoref{poly-size}) in polynomial time, we can
		also solve \gb on $G$ in polynomial time.
	\end{proof}

	\subsection{\mbpt is \NP-Complete}

	We now consider the original \mbpt problem. As mentioned in
	\autoref{1-preliminaries} this is equivalent to partitioning the edges of
	a graph. We would like to immediately draw an equivalence between \mbpt
	and \geb, but note that for a matrix $M$ the associated graph $G(M)$ is
	always bipartite. However, we can resolve this:

	\begin{definition}
		Given a graph $G = (V, E)$, it's \textit{edge split graph}
		$\mathcal{S}(G) = (V', E')$ is given as:

		$$V' = V \cup \{\, v_e \,\mid\, e \in E \,\}$$
		$$E' = \bigcup_{e = \{u, w\} \in E} \{\,\{u, v_e\}\, \{v_e, w\}\,\}$$
	\end{definition}

	In other words, we replace each edge by a path of length two. The resulting
	graph is clearly bipartite (with sides $V$ and $V' \setminus V$). Using
	this bipartite extension of a graph, we can build a matrix and use \mbpt
	to solve \geb. This process is displayed in figure \ref{}.

	\todo{Insert picture}

	First, we prove that we can safely take the \textit{edge split graph}
	without affecting the \geb problem.

	\begin{proposition}
		\label{split-graph}
		For any graph $G$ we have $\sgeb(G) = \sgeb(\mathcal{S}(G))$.
	\end{proposition}

	\begin{proof}
		Let $G = (V, E)$ and $\mathcal{S}(G) = (V', E')$.
		\begin{enumerate}
			\item[$(\geq)$] Let $\pi$ be an optimal coloring of $E$. We define
					a coloring $\pi'$ of $E'$ as follows. Let
					$e' = \{u, v_e\} \in E'$ with $u \in V \subseteq V'$ and
					$v_e \in V' \setminus V$. So $e'$ is half of the length two
					path associated with $e$ in $G$. We set
					$\pi'(e') = \pi(e)$, that is, we give each edge in $E'$ the
					color of the edge in $E$ that induced it.

					Then no vertices in $V'\setminus V$ were colored, since
					both edges in a path have the same color, and the vertices
					cut by $\pi'$ in $V \subseteq V'$ are precisely those
					cut by $\pi$ in 
			\item[$(\leq)$] abcd
		\end{enumerate}
	\end{proof}
\end{document}
