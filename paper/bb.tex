\documentclass{article}

\usepackage[a4paper, margin=1.37in]{geometry}
\usepackage[english]{babel}
\usepackage[utf8]{inputenc}
\usepackage{amsmath}
\usepackage{amsfonts}
\usepackage{amsthm}
\usepackage{amssymb}
\usepackage{fancyhdr}
\usepackage{xifthen}
\usepackage{tabularx,lipsum,environ,amsmath,amssymb}
\usepackage{cite}
\usepackage[T1]{fontenc}
\usepackage{todonotes}
\usepackage{hyperref}
\usepackage{tikz}
\usepackage{aliascnt}
\usepackage{tikz-qtree,tikz-qtree-compat}

\usetikzlibrary{shapes,arrows,matrix,positioning,calc,positioning,automata,shadows,fit}

% Matrix drawing
\newcommand{\msize}{2em}
\tikzstyle{redblock} = [draw, fill=red, rectangle, minimum width=\msize, minimum height=\msize]
\tikzstyle{blueblock} = [draw, fill=blue, rectangle, minimum width=\msize, minimum height=\msize]
\tikzstyle{grayblock} = [draw, fill=gray, rectangle, minimum width=\msize, minimum height=\msize]
\tikzstyle{whiteblock} = [draw, fill=white, rectangle, minimum width=\msize, minimum height=\msize]

\newcommand{\redb}{\node[redblock] {};}
\newcommand{\blueb}{\node[blueblock] {};}
\newcommand{\grayb}{\node[grayblock] {};}
\newcommand{\whiteb}{\node[whiteblock] {};}
	
%	\begin{tikzpicture}
%	\matrix[matrix of nodes] {
%		\redb  &\redb  &\whiteb&\whiteb\\
%		\redb  &\redb  &\whiteb&\blueb \\
%		\whiteb&\whiteb&\blueb &\blueb \\
%		\whiteb&\whiteb&\blueb &\whiteb\\
%	};
%	\end{tikzpicture}

% NP PROBLEM
\makeatletter
\newcommand{\problemtitle}[1]{\gdef\@problemtitle{#1}}% Store problem title
\newcommand{\probleminput}[1]{\gdef\@probleminput{#1}}% Store problem input
\newcommand{\problemquestion}[1]{\gdef\@problemquestion{#1}}% Store problem question
\NewEnviron{problem}{
  \problemtitle{}\probleminput{}\problemquestion{}% Default input is empty
  \BODY% Parse input
  \par\addvspace{.5\baselineskip}
  \noindent
  \begin{tabularx}{\textwidth}{@{\hspace{\parindent}} l X c}
    \multicolumn{2}{@{\hspace{\parindent}}l}{\textsc{\@problemtitle}} \\% Title
    \textbf{Input:} & \@probleminput \\% Input
    \textbf{Question:} & \@problemquestion% Question
  \end{tabularx}
  \par\addvspace{.5\baselineskip}
}

% THEOREM ENVIRONMENTS
\newtheorem{theorem}{Theorem}[section]

\newaliascnt{lemma}{theorem}
\newtheorem{lemma}[lemma]{Lemma}
\aliascntresetthe{lemma}
\providecommand*{\lemmaautorefname}{Lemma}

\newaliascnt{proposition}{theorem}
\newtheorem{proposition}[proposition]{Proposition}
\aliascntresetthe{proposition}
\providecommand*{\propositionautorefname}{Proposition}

\newaliascnt{corollary}{theorem}
\newtheorem{corollary}[corollary]{Corollary}
\aliascntresetthe{corollary}
\providecommand*{\corollaryautorefname}{Corollary}

\newaliascnt{definition}{theorem}
\newtheorem{definition}[definition]{Definition}
\aliascntresetthe{definition}
\providecommand*{\definitionautorefname}{Definition}

%\newtheorem{lemma}[theorem]{Lemma}
%\newtheorem{proposition}[theorem]{Proposition}
%\newtheorem{corollary}[theorem]{Corollary}
%\newtheorem{definition}[theorem]{Definition}

% NP Complete
\newcommand{\NP}{$\mathcal{NP}$}
\newcommand{\NP}{$\mathcal{P}$}

% Do not split footnotes over multiple pages
\interfootnotelinepenalty=10000




\begin{document}

	\section{Exact Algorithm}

	In this section we give an exact algorithm for finding an exact
	bipartitioning of a matrix, extending the branch and bound
	algorithm by Pelt \&
	Bisseling \cite{pelt-and-bisseling}. We first give a quick overview
	of their approach, a better understanding can be gained by reading the
	paper itself.

	After that we give two new bounds that extend the reach of the previous
	algorithm.

	\subsection{Previous work}

	All of this subsection concerns work done in \cite{pelt-and-bisseling}.

	Recall that a branch and bound algorithm initially starts with
	(a representation of) the whole solution space, and then repeatedly
	branches on properties of the solutions until these are refined
	enough that they specify a single solution (this is a leaf in the branch
	and bound tree). When the properties are
	chosen carefully, we may cut off (`bound') large parts of the search
	tree.

	In the case of matrix bipartitioning, a first obvious choice would 
	be to branch on
	which side of the partition we put each nonzero in. For an $n \times m$
	matrix with $N$ nonzeros this results in $2^N$ leaf nodes. However,
	\todo{waarom niet?, even goed praten waarom dit een slecht idee is}.
	Instead, we can branch of the status of each of the
	rows and columns of the matrix: each of them is either entirely red,
	entirely blue, or `cut', i.e. it contains both colors. As a result, we
	only have $3^{n+m}$ leafs, which is already smaller than $2^N$ when
	$n + m < \log_3(2) N \approx 0.63N$. However, not all of the $3^{n+m}$
	states are even reachable: if a row and column intersect in a nonzero,
	we cannot mark one of them as \textit{red} and one of them as \textit{blue}
	(i.e. we do require assignments be consistent).

	When we traverse the branch and bound tree, at each stage we have a
	`partial assignment', where some of the rows and columns are
	\textit{red}, \textit{blue} or \textit{cut}, and some are still
	unassigned. For a given matrix $M$ and its bipartite graph representation
	$G(M) = (V, E)$ (recall \autoref{}),
	we will write $R \subseteq V$ (resp. $B, C \subseteq V$)
	for the vertices (corresponding to rows and columns) that were assigned
	\textit{red} (resp. \textit{blue}, \textit{cut}). Additionally, while
	all remaining vertices are unassigned, they may still be connected to
	vertices in $R, B$ and $C$. For example, if an unassigned column
	vertex $u$ is
	adjacent to a row vertex in $r \in R$, this means that $M_{ru}$ is nonzero.
	In particular, since row $r$ is red, we cannot make $u$ blue. So we will
	call $u$ \textit{partially red}, written as $P_R$ (with $P_B$ defined
	analogously). Finally, an unassigned vertex may have neighbours in both
	$R$ and $B$. While unassigned, we are basically forced to cut this vertex.
	Because of this, we assume that whenever such a vertex is created, we
	immediately cut it. So we will ignore these vertices.

	For pruning, the following two bounds are used:
	\begin{enumerate}
		\item[] \textbf{Packing bound} Let $E(R)$ denote all the edges that
				are colored red by the current partial assignment (that is, all
				edges adjacent to a vertex in $R$).
				For each partially red vertex
				$p \in P_R$ (i.e. $p$ has a vertex in $R$ adjacent), let
				$$N^R(p) = \{\, e \in E \,\mid\, \text{$e$ is adjacent to $p$
				but not to $R$ or $B$}\,\}$$
				Then all edges in $N^R(p)$ are still free, but directly
				adjacent to a red edge, since $p$ is adjacent to some $r\in R$,
				so we are forced to color $\{p, r\}$ red.

				If we don't want to cut any more vertices, we have to color all
				edges in $N^R(p)$ red. But if
				$|E(R)| + \sum_{p\in P_R} |N^R(p)| > \frac{1}{2}|E|$, this will
				lead to an unbalanced partition, and we are forced to cut some
				of the vertices $p \in P_R$. Since we are looking for a lower
				bound, we can greedily take those vertices with largest
				$|N^R(p)|$ until the sum is small enough again.

				Note that we implicitly assumed all the $N^R(p)$ were disjoint,
				so we can assign their edges independently. Since
				$G(M) = (V,E)$ is bipartite, this is true if we consider each
				side of the bipartition separately. We can then do the same
				for $P_B$, and add everything together to get the
				\textit{packing bound}.
		\item[] \textbf{Matching bound} Let's consider some $p \in P_R$ and
				$q \in P_B$ with $\{p, q\} \in E$. We note that $p$ has a red
				edge adjacent (through its adjacent vertex in $R$), and $q$ has
				a blue edge adjacent. Clearly no matter what color we give
				$\{p, q\}$, we will have to cut one of them.

				We can improve this by finding a maximal set of such edges that
				are disjoint (this is necessary, since otherwise we could
				resolve $\{p, q\}$ and $\{p, q'\}$ just by cutting $p$).
				The relevant graph is just our bipartite graph
				$G(M) = (V, E)$ restricted to
				$P_R \cup P_B \subseteq V$, keeping only edges with one
				endpoint in $P_R$ and one in $P_B$. Bipartite
				matching is a classical problem we can solve in polynomial
				time. Then the size of the maximum matching is a lower bound
				on the number of vertices that still have to be cut by any
				extension of the current partial assignment.
	\end{enumerate}

	These two bounds conflict with each other, so must take the maximum of the
	two. We add to this the total number of already cut vertices $|C|$ to
	obtain a lower bound on any extension of the current partial assignment.
	If this matches the cost of the best solution already seen (an obvious
	upper bound on the optimal solution), we can prune the current subtree.

	One issue with these two bounds is that they are in some sense `local'
	bounds. They both only consider the direct neighbourhoods of $R$ and $B$
	(through $P_R$ and $P_B$). The two bounds we give next can be seen as
	extending the above bounds to the whole graph, taking full advantages of
	its connectivity.

	\subsection{Flow Bound}

	We will begin 

	\subsection{Extended Packing Bound}

	\subsection{Further implementation notes}

\end{document}
