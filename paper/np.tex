\documentclass{article}

\usepackage[a4paper, margin=1.37in]{geometry}
\usepackage[english]{babel}
\usepackage[utf8]{inputenc}
\usepackage{amsmath}
\usepackage{amsfonts}
\usepackage{amsthm}
\usepackage{amssymb}
\usepackage{fancyhdr}
\usepackage{xifthen}
\usepackage{tabularx,lipsum,environ,amsmath,amssymb}
\usepackage{cite}
\usepackage[T1]{fontenc}
\usepackage{todonotes}
\usepackage{hyperref}
\usepackage{tikz}
\usepackage{aliascnt}
\usepackage{tikz-qtree,tikz-qtree-compat}
\usepackage{url}
\usepackage{graphicx}

\usetikzlibrary{shapes,arrows,matrix,positioning,calc,positioning,automata,shadows,fit}

% Matrix drawing
\newcommand{\msize}{2em}
\tikzstyle{redblock} = [draw, fill=red, rectangle, minimum width=\msize, minimum height=\msize]
\tikzstyle{blueblock} = [draw, fill=blue, rectangle, minimum width=\msize, minimum height=\msize]
\tikzstyle{grayblock} = [draw, fill=gray, rectangle, minimum width=\msize, minimum height=\msize]
\tikzstyle{whiteblock} = [draw, fill=white, rectangle, minimum width=\msize, minimum height=\msize]

\newcommand{\redb}{\node[redblock] {};}
\newcommand{\blueb}{\node[blueblock] {};}
\newcommand{\grayb}{\node[grayblock] {};}
\newcommand{\whiteb}{\node[whiteblock] {};}
	
%	\begin{tikzpicture}
%	\matrix[matrix of nodes] {
%		\redb  &\redb  &\whiteb&\whiteb\\
%		\redb  &\redb  &\whiteb&\blueb \\
%		\whiteb&\whiteb&\blueb &\blueb \\
%		\whiteb&\whiteb&\blueb &\whiteb\\
%	};
%	\end{tikzpicture}

% NP PROBLEM
\makeatletter
\newcommand{\problemtitle}[1]{\gdef\@problemtitle{#1}}% Store problem title
\newcommand{\probleminput}[1]{\gdef\@probleminput{#1}}% Store problem input
\newcommand{\problemquestion}[1]{\gdef\@problemquestion{#1}}% Store problem question
\NewEnviron{problem}{
  \problemtitle{}\probleminput{}\problemquestion{}% Default input is empty
  \BODY% Parse input
  \par\addvspace{.5\baselineskip}
  \noindent
  \begin{tabularx}{\textwidth}{@{\hspace{\parindent}} l X c}
    \multicolumn{2}{@{\hspace{\parindent}}l}{\textsc{\@problemtitle}} \\% Title
    \textbf{Input:} & \@probleminput \\% Input
    \textbf{Question:} & \@problemquestion% Question
  \end{tabularx}
  \par\addvspace{.5\baselineskip}
}

% THEOREM ENVIRONMENTS
\newtheorem{theorem}{Theorem}[section]

\newaliascnt{lemma}{theorem}
\newtheorem{lemma}[lemma]{Lemma}
\aliascntresetthe{lemma}
\providecommand*{\lemmaautorefname}{Lemma}

\newaliascnt{proposition}{theorem}
\newtheorem{proposition}[proposition]{Proposition}
\aliascntresetthe{proposition}
\providecommand*{\propositionautorefname}{Proposition}

\newaliascnt{corollary}{theorem}
\newtheorem{corollary}[corollary]{Corollary}
\aliascntresetthe{corollary}
\providecommand*{\corollaryautorefname}{Corollary}

\newaliascnt{definition}{theorem}
\newtheorem{definition}[definition]{Definition}
\aliascntresetthe{definition}
\providecommand*{\definitionautorefname}{Definition}

%\newtheorem{lemma}[theorem]{Lemma}
%\newtheorem{proposition}[theorem]{Proposition}
%\newtheorem{corollary}[theorem]{Corollary}
%\newtheorem{definition}[theorem]{Definition}

% NP Complete
\newcommand{\NP}{$\mathcal{NP}$}

% Do not split footnotes over multiple pages
\interfootnotelinepenalty=10000

% Decision problems
\newcommand{\mbpt}{\textsc{Matrix \allowbreak Bipartition} }
\newcommand{\gb}{\textsc{Graph \allowbreak Bisection} }
\newcommand{\geb}{\textsc{Graph \allowbreak Edge-\allowbreak Bisection} }
\newcommand{\bgeb}{
	\textsc{Bipartite \allowbreak Graph \allowbreak Edge-\allowbreak Bisection}
 }
\newcommand{\sgeb}{\textsc{GEB} }
\newcommand{\sgb}{\textsc{GB} }


% Add commands here

% Decision problems
\newcommand{\mbpt}{\textsc{Matrix Bipartition}}
\newcommand{\gb}{\textsc{Graph Bisection}}
\newcommand{\geb}{\textsc{Graph Edge-Bisection}}
\newcommand{\bgeb}{\textsc{Bipartite Graph Edge-Bisection}}
% /commands

\begin{document}

	\section{Hardness results}
	In this section we will formally analyze the matrix partitioning
	problem and prove that it is \NP-Complete, even if we fix the
	the number of processors to $k = 2$. We will assume we are looking
	for a perfect partitioning, i.e. with imbalance parameter $\epsilon = 0$.

	\subsection{Preliminaries}
	To begin, let us define a formal decision-variant of the matrix
	partitioning problem for $k = 2$, based on the optimization
	variant described in \ref{introductory-section} where the goal
	is to minimize the total communication volume.

	\begin{problem}
		\problemtitle{\mbpt}
		\probleminput{A $n\times m$ matrix $M$, whose nonzeros are precisely
			indexed by the set
			$Z \subseteq \{\, 1, \dots, n \,\} \times
				\{\, 1, \dots, m \,\}$, and an integer $V$, the required
			maximum volume.}
		\problemquestion{Does there exist a disjoint partitioning of $Z$
			into $Z_1 \cup Z_2$ such that $|Z_1| = |Z_2|$ and volume
			$VOL(Z_1, Z_2) \leq V$?}
	\end{problem}

	Here $VOL(Z_1, Z_2)$ counts the number of rows and columns that
	have nonzeros in $Z_1$ and $Z_2$, as before.

	When thinking about the matrix bipartitioning problem, it is helpful
	to reformulate it in terms of graphs. Given an $n \times m$ matrix $M$
	we can define a bipartite adjacency graph $G(M) = (V(M), E(M))$ with
	$n$ vertices representing the rows of $M$, and $m$ vertices representing
	the columns, where a row vertex $r$ and a column vertex $c$ are connected
	if and only if $M_{rc}$ is nonzero.

	This equivalence extends to the partitioning problem. A bipartitioning of
	the nonzeros of $M$ corresponds to a bipartitioning of the edges of $G(M)$,
	and the rows and columns contributing to the final volume are correspond
	precisely to the vertices with edges in both sides of the partition. See
	also figure \ref{figure-for-graph-equiv}

	To this end 

		
\end{document}
